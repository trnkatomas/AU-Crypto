% author: Tomas Trnka
% mail: tomas@trnkatomas.eu
% date: 2013-07-04

\documentclass[a4paper,10pt]{article}
%\usepackage[czech]{babel}
%\usepackage[T1]{fontenc}
\usepackage[hmargin=2.2cm,vmargin=2.2cm]{geometry}
\usepackage[utf8x]{inputenc}
\usepackage{fancyhdr}
\usepackage{amsmath} 
\usepackage{dsfont}
\usepackage{tikz}
\usetikzlibrary{patterns}
\usetikzlibrary{calc}
\usepackage{enumerate}
\pagestyle{fancy}
\headheight 15pt
\lhead{Crpyto, Fall 2014}
\rhead{Tomas Trnka}
\def\firstcircle{(270:1.75cm) circle (2.5cm)}
\def\secondcircle{(150:1.75cm) circle (2.5cm)}
\def\thirdcircle{(30:1.75cm) circle (2.5cm)}
\begin{document}
\section*{Public cryptography -- Exercise 6}
The goal of the assignment is to prove that the nice property that holds for $\mathds{Z}^*_N$ 
$$
D(E(m)^d) = D(m^{e^d}) = D(m^{ed}) = m
$$
is also true in noncyclic group $\mathds{Z}_N$.

\noindent
From the Stinsons's book we know that:
$$
\begin{array}{c}
n = pq\\
p,q\text{ are primes and } p \neq q\\
ab \equiv 1 \mod (p-1)(q-1)\\
e(x) \equiv x^b \mod n\\
d(y) \equiv x^a \mod n\\
\text{and the results following from the Chineese reminder teorem}\\
x_1 \equiv x_2 \mod (pq) \leftrightarrow (x_1 \equiv x_2 \mod p \wedge x_1 \equiv x_2 \mod q)
\end{array}
$$

\noindent
While decoding we would like to obtain:
$$
d(y) \equiv d(e(x)^a) \equiv x^{b^a} \equiv x \mod n\\
$$

\noindent
But even in the $\mathds{Z}_N$ world holds that:
$$
x^{ab} = x^{1+k(p-1)(q-1)}
$$
because the number $p$, $q$ are primes. Now we have to decide which one from the $p$ and $q$ will be divider of the $x$. We can assume that the $q$ will be divider.

\noindent
Let's examine the part for the $p$:
$$
\begin{array}{c}
x^{ab} \equiv x^{1+k(p-1)(q-1)}\\
x^{1+k(p-1)(q-1)} \equiv x \cdot \left( x^{(p-1)} \right) ^{k(q-1)}\\
\end{array}
$$
And since the $p$ is a prime we can derive from the Fermat little theorem that:
$$
x \cdot \left( x^{(p-1)} \right) ^{k(q-1)} \equiv x \cdot 1^{k(q-1)} \mod p
$$
So we have successfully proved that 
$$
x^{b^a} \equiv x \mod p\\
$$

\noindent
The next part is even easier, if $x$ is divisible by $q$ then must hold that:
$$
x \mod q \equiv 0
$$

\noindent
The result of previous equation is that:
$$
x^{ab} \equiv 0^{ab} \equiv 0 \mod q
$$

\noindent
We have proved that $x_1 \equiv x_2 \mod p$ is true and $x_1 \equiv x_2 \mod q$ alike. Therefore the conditions for the $x_1 \equiv x_2 \mod (pq)$ hold and we can claim that the nice property for decryption holds also in the $\mathds{Z}_N$ world.
%\begin{enumerate}[a)]
%\end{enumerate}
\end{document}