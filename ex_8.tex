% author: Tomas Trnka
% mail: tomas@trnkatomas.eu
% date: 2013-07-04

\documentclass[a4paper,10pt]{article}
%\usepackage[czech]{babel}
%\usepackage[T1]{fontenc}
\usepackage[hmargin=2.2cm,vmargin=2.2cm]{geometry}
\usepackage[utf8x]{inputenc}
\usepackage{fancyhdr}
\usepackage{amsmath} 
\usepackage{dsfont}
\usepackage{hyperref}
\usepackage{algpseudocode}
\usepackage{tikz}
\usetikzlibrary{patterns}
\usetikzlibrary{calc}
\usepackage{enumerate}
\pagestyle{fancy}
\headheight 15pt
\lhead{Crpyto, Fall 2014}
\rhead{Tomas Trnka}
\def\firstcircle{(270:1.75cm) circle (2.5cm)}
\def\secondcircle{(150:1.75cm) circle (2.5cm)}
\def\thirdcircle{(30:1.75cm) circle (2.5cm)}
\begin{document}
\section*{Multiplication algorithm -- Exercise 8}
\begin{enumerate}[a)]
\item Showing that in every step we have multiplied $Z_i$ is straightforward when we realized what the algorithm does in the binary numeral system,

\begin{enumerate}[I.]
\item While running our algorithm the $x$ remains $1$ until we encounter the left most $1$ in the binary notation of $z$, when the first $1$ occurs $x$ becomes $a$.

\item Then in every following step the $x$ is squared. It holds that $2=10_b$ in binary notation. Let's progress with a little bit untidy notation where we have $x$ in decimal and the exponent in binary.
$$
x^{2} = (a^{{Z_i}_b})^{10_b}
$$
But what happens when we multiply in binary notation by $2=10_b$? We are just shifting the whole binary number one position to the left while adding zero at the end.

\item Then if the next number of the binary notation for $Z$ was zero we are done and we can skip the step $b$ and continue with looping. But if it was $1$, we have to multiply our number by another $a$.

Which in binary is again:
$$
a^{{Z_i}_b}\cdot a^{1_b} = a^{{Z_i}_b+1}
$$

But from the previous step we know that after the step $a$ the exponent in binary notation ends with zero. So such addition just switch the last bit to $1$. We can clearly see that we have moved one position to the right in our number $z$ and we adjusted the $x$ accordingly to the last seen bit.
\end{enumerate}

\item We can expect for the $k$ iterations that we have to compute $k$-times the step $2a$. On the other hand when we assume that the bits in binary notation of $z$ are distributed randomly can expect that we have to compute step $2b$ only in half the $k$ iterations. The expected number of iterations therefore will be $k + \frac{k}{2}$.

\item The algorithm can by improved by scanning for the next two consecutive bits. We have to precompute values for $a^2$ and $a^3$ before we start looping over the $Z$.
In the step $2a$ we have to set the value for the $x$ as $x^4$, because in binary notation it would mean shifting by two positions to the left. In our new step $2b$ we have to recognize which one of the precomputed values we have to apply according to the next 2 bits. The pseudo code is as follows:
\begin{algorithmic}
\State $x\gets 1$
\For{$i\gets k-1$ downto $0$ by $2$}
\State $x\gets x^4$
\If{$z_iz_{i-1}==11$}
	\State $x\gets x \cdot a^3$
\ElsIf{$z_iz_{i-1}==10$}
	\State $x\gets x \cdot a^2$	
\ElsIf{$z_iz_{i-1}==01$}
	\State $x\gets x \cdot a$
\Else
	\State $x \gets x$ // i.e. do nothing
\EndIf
\EndFor
\end{algorithmic}
Then when we count all the multiplication we have to do the number of operations will be:
$$
2 + \frac{k}{2} + \frac{k}{2}\cdot\frac{3}{4}
$$
We have precomputed 2 values, then we have only half of the $k$ iterations and in only one from the 4 variants $b$ we do nothing.

\item 
This idea can be generalized and the general formula for scanning would be:
$$
(2^m-2) + \frac{k}{m} \left( 1 + \frac{2^m - 1}{2^m} \right)
$$
where $m$ is the number of bits that we are scanning ahead.

We can clearly see that this function depends on the size of $k$. We can derive the formula by the variable $n$ and obtain:
$$
-\frac{k \left(2^{-n} \left(2^n-1\right)+1\right)}{n^2}+\frac{k \left(\log (2)-2^{-n} \left(2^n-1\right) \log (2)\right)}{n}+2^n \log (2)
$$

When we know the value of $k$ we can now put it into the formula, determine when the formula equals to zero and obtain the optimal value for $m$, i.e. how many bits will be optimal to precompute ahead.

\begin{figure}
\centering
\includegraphics[scale=0.7]{example.png}
\caption{Graph for values of $k=2000$ where optimal value of $m$ is 7}
\end{figure}
\end{enumerate}
\end{document}