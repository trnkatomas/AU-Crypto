% author: Tomas Trnka
% mail: tomas@trnkatomas.eu
% date: 2013-07-04

\documentclass[a4paper,10pt]{article}
%\usepackage[czech]{babel}
%\usepackage[T1]{fontenc}
\usepackage[hmargin=2.2cm,vmargin=2.2cm]{geometry}
\usepackage[utf8x]{inputenc}
\usepackage{fancyhdr}
\usepackage{fancyvrb}
\usepackage{amsmath} 
\usepackage{float}
\usepackage{enumerate}
\usepackage{tikz}
\usepackage{hyperref}
\pagestyle{fancy}
\headheight 15pt
\lhead{Crpyto, Fall 2014}
\rhead{Tomas Trnka}
\newcommand{\set}[1]{\ensuremath{\left\lbrace #1 \right\rbrace}}
\newcommand{\role}[1]{\ensuremath{\left\langle #1 \right\rangle}}
\newcommand{\cara}{\begin{center}\rule{140mm}{.2mm}\end{center}}
\newcommand{\mI}{\ensuremath{^\mathcal{I}}}
\newcommand{\Tbox}[1]{\ensuremath{\mathcal{T}}-Box#1}
\newcommand{\Abox}[1]{\ensuremath{\mathcal{A}}-Box#1}
\newcommand{\mC}[1]{\ensuremath{\mathcal{#1}}}
\newcommand{\Tc}{\ensuremath{\mathcal{T}_c}}
\newcommand{\qb}[1]{\ensuremath{\vert{#1}\rangle}}
\begin{document}
\section*{Public-key crypto based on discrete logarithms}
\begin{itemize}
\item[] \textbf{The discrete log (DL) problem}\\
Given a group $G$, generator $\alpha$, and $\beta \in G$, find integer $a$, such that $\alpha^a = \beta$. The DL problem is in many groups notoriously hard, for instance in $Z^{∗}_p$.
\item[]\textbf{The Diffie-Hellman (DH) problem}\\
Given a group $G$, generator $\alpha$, and $\alpha^a,\alpha^b$, where $a,b$ are randomly and independently chosen from $Z_t$, compute $\alpha^{ab}$.
Clearly, if we could find a from $\alpha^a$, we could solve DH by a single exponentiation, so the DH problem is no harder than the DL problem. (polynomial reduction)
\item[] \textbf{The Decisional Diffie-Hellman (DDH) problem}\\
Given a group $G$, generator $\alpha$, and $\alpha^a,\alpha^b,\alpha^c$, where $a,b$ are randomly and independently chosen from $Z_t$; and where $c$ is chosen either as $c = ab$, or uniformly random from $Z_t$. Now guess which of the two cases we are in. Clearly, if you could solve DH, then you could solve DDH, by computing $\alpha{ab}$ and comparing this to $\alpha^c$. So we can say that DDH problem is no harder than the DH problem. (again reduction)
\end{itemize}

\end{document}